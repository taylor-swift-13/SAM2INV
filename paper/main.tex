%%
%% SAM2INV: A Sampling-and-Filtering Framework with Offline Policy Exploration
%%          for Certified Loop Invariant Generation
%% ACM Conference Format
%%
\documentclass[sigconf,nonacm]{acmart}

%% Packages
\usepackage{booktabs}
%% algorithm/algorithmic not available in this TeX installation;
%% using a listing-based pseudocode style instead.
\usepackage{listings}
\usepackage{xcolor}
\usepackage{subcaption}
\usepackage{amsmath}
\usepackage{amssymb}
\usepackage{pifont}

%% Listings style for C / ACSL
\lstdefinestyle{cstyle}{
  language=C,
  basicstyle=\ttfamily\small,
  keywordstyle=\color{blue}\bfseries,
  commentstyle=\color{gray}\itshape,
  stringstyle=\color{red},
  numbers=left,
  numberstyle=\tiny\color{gray},
  stepnumber=1,
  numbersep=5pt,
  backgroundcolor=\color{white},
  showspaces=false,
  showstringspaces=false,
  showtabs=false,
  frame=single,
  tabsize=2,
  captionpos=b,
  breaklines=true,
  breakatwhitespace=false,
  escapeinside={(*@}{@*)},
  morekeywords={loop,invariant,assigns,variant,requires,assert},
  literate={\\at}{{{\color{blue}\textbackslash at}}}3
           {\\forall}{{{\color{blue}\textbackslash forall}}}7
           {\\exists}{{{\color{blue}\textbackslash exists}}}7,
}
\lstset{style=cstyle}

%% Listings style for Coq
\lstdefinestyle{coqstyle}{
  language=,
  basicstyle=\ttfamily\small,
  keywordstyle=\color{blue}\bfseries,
  commentstyle=\color{gray}\itshape,
  numbers=left,
  numberstyle=\tiny\color{gray},
  frame=single,
  breaklines=true,
  morekeywords={Definition,forall,Proof,Qed,Lemma,Theorem,Require,Import},
}

%% Meta
\title{A Sampling-and-Filtering Framework with Offline Policy Exploration for Certified Loop Invariant Generation}

\author{Anonymous}
\affiliation{%
  \institution{Anonymous Institution}
}
\email{anonymous@example.com}

\begin{document}

%% Abstract
\begin{abstract}
Automatically synthesizing loop invariants remains a central challenge in formal software
verification.
We present \textsc{SAM2INV}, a system built around three contributions.
First, \textsc{LoopFactory}, a probabilistic DSL that generates structured numeric loop
programs with controllable arithmetic complexity, nesting depth, and variable count,
serves as the \emph{task input generator}: it supplies the programs that the downstream
pipeline annotates and that ultimately populate the training corpora.
Second, a \emph{three-stage sampling-and-filtering pipeline} combines large language
model~(LLM) candidate generation with progressively more expensive filters to produce
\emph{certified} ACSL loop invariant annotations for C programs, formally verified by
Frama-C/WP.
$N$ parallel LLM calls are issued and all returned candidate expressions are
\emph{unioned} into a single pool, which is then filtered through
(1)~an \emph{ACSL syntax filter} that rejects ill-formed expressions,
(2)~an \emph{execution-trace semantic filter} that discards candidates falsified by any
observed loop state, and
(3)~\emph{Houdini formal pruning} via Frama-C/WP that removes non-inductive candidates,
converging to the strongest certified inductive conjunction the union supports.
Crucially, the filtering pipeline alone---before any fine-tuning---already yields
substantial improvements in success rate and invariant quality over a single-call baseline.
Third, the pipeline naturally produces \emph{offline SFT and DPO training data} at no
extra cost: every certified annotation becomes an SFT record, and every rejected
candidate---labelled by the stage at which it was filtered---forms a DPO preference pair.
This stage-stratified labelling avoids reward hacking and requires no online
reinforcement learning.
Running \textsc{LoopFactory} and this pipeline together at scale constitutes the training
data generation workflow.
\end{abstract}

\maketitle

%% ============================================================
%% ============================================================
\section{Introduction}\label{sec:intro}
%% ============================================================

Loop invariants are the cornerstone of deductive program verification.
Given a loop annotated with an appropriate invariant, a verification condition generator
such as Frama-C/WP~\cite{framac} can reduce the correctness of the entire program to
a set of first-order proof obligations dischargeable automatically.
In practice, however, \emph{finding} the right invariant is the bottleneck:
even expert users spend substantial effort crafting invariants that are simultaneously
\emph{inductive} (preserved by each iteration), \emph{sufficient} (strong enough to
imply the postcondition), and \emph{syntactically admissible} (expressible in the
annotation language).

Classical approaches fall into two broad camps.
\emph{Static} techniques such as abstract interpretation~\cite{cousot1977abstract},
template-based constraint solving~\cite{colon2003linear}, and
Houdini-style iterative weakening~\cite{flanagan2001houdini}
offer soundness guarantees but are confined to fixed abstract domains or templates.
\emph{Dynamic} techniques infer candidates from execution traces using polynomial
fitting~\cite{daikon}, machine learning~\cite{si2018learning}, or data-driven
enumerative search~\cite{garg2016learning}, but must still close the gap between
empirical observation and formal proof.

Recently, large language models~(LLMs) have demonstrated remarkable ability at code
understanding and generation.
Several studies have explored LLM-driven invariant
synthesis~\cite{chakraborty2024towards,pei2023can,kamath2023finding},
showing that LLMs can produce plausible candidates even for non-trivial programs.
However, existing approaches suffer from two structural weaknesses.
First, they rely on a single LLM call or a simple repair loop without \emph{systematic
multi-stage filtering}, so many returned candidates are syntactically ill-formed,
semantically inconsistent with program behaviour, or formally non-inductive.
Second, the verification signal produced by these systems is discarded after each
run; no mechanism exists for using that signal to \emph{improve} the LLM policy
that generates the candidates.
The result is that each new program is treated as if the model had learned nothing
from prior successes and failures.

\paragraph{Why invariant quality is hard to learn from.}
A fundamental obstacle to improving LLM-based invariant generation
through reinforcement learning is that
\emph{evaluating candidate quality is structurally different here
from other code generation tasks}.
The correct annotation for a loop is not a single program but a
\emph{conjunction} of multiple invariant clauses, each capturing a
different arithmetic relationship.
This has two consequences.
First, the binary verifier signal is \emph{non-attributable}:
the final correct annotation is typically assembled from contributions
of several parallel LLM calls, so no individual call can be scored
as simply correct or incorrect.
Second, the binary signal is susceptible to \emph{reward hacking}:
trivially weak annotations (e.g., \lstinline|loop invariant true;|)
can formally pass a checker on programs with loose postconditions
yet carry no mathematical insight about the loop.
These two properties together explain why online reinforcement
learning methods such as
GRPO~\cite{deepseekr1}---which score individual completions against
each other via binary verifier feedback---are poorly suited to this
domain.

\paragraph{Our approach.}
We present \textsc{SAM2INV} (\textbf{Sam}pling \textbf{to}
\textbf{Inv}ariant), a system that addresses both problems through a
two-stage pipeline: first generate a large, diverse corpus of loop
programs; then for each program, generate, filter, and certify
invariant candidates---logging every accepted and rejected expression
as structured training data.

\textbf{Contribution~1: \textsc{LoopFactory}---a Probabilistic DSL for Task Input Generation.}
Existing benchmarks contain only a few dozen loop programs, which
is insufficient for training.
We design \textsc{LoopFactory}, a probabilistic domain-specific
language~(DSL) that generates structured numeric loop programs with
tunable arithmetic complexity (linear vs.\ nonlinear), nesting depth,
variable count, and loop-control mode.
\textsc{LoopFactory} is the \emph{task input generator}: it produces
the programs that the invariant-generation pipeline annotates, and
hence the programs that ultimately populate the SFT and DPO training
corpora.
Without a scalable supply of programs with known structure, offline
training data cannot be generated at the scale needed for fine-tuning.

\textbf{Contribution~2: A Three-Stage Sampling-and-Filtering Pipeline.}
Given a program (from \textsc{LoopFactory} or any source),
\textsc{SAM2INV} issues $N$ parallel LLM calls and \emph{unions} all
returned candidate invariant expressions into a single pool, rather
than selecting among them.
The pool is then filtered through three progressively more expensive
stages:
(i)~an ACSL \emph{syntax filter} that rejects ill-formed expressions
without running any code;
(ii)~an \emph{execution-trace semantic filter} that discards candidates
falsified by any observed loop state;
and (iii)~\emph{Houdini formal pruning} that uses Frama-C/WP to
remove non-inductive candidates, converging to the strongest
inductive conjunction the union supports.
Every survivor is \emph{certified} by Frama-C/WP.
Importantly, \emph{the filtering pipeline alone---before any
fine-tuning---already yields substantial improvements in both
success rate and invariant quality} over a single-call baseline.
Union assembly makes the final annotation stronger by combining
conjuncts that different calls discover independently;
the syntax and trace filters ensure that only well-formed,
semantically consistent candidates reach Frama-C, eliminating
parse failures and reducing the cost of Houdini pruning;
and Houdini finds the maximal inductive conjunction the
pre-screened union can support.
The pipeline is therefore a strong standalone tool for invariant
generation, and the offline training data collection is an
additional improvement layer built on top of it.
Running \textsc{LoopFactory} and this pipeline together at scale
constitutes the training data generation workflow: each program
yields one certified annotation and a set of stage-labelled rejected
candidates.

\textbf{Contribution~3: Offline SFT and DPO Data from Pipeline Logs.}
The union-then-filter design makes online RL inapplicable for the
reasons above, but it naturally produces structured training data
\emph{at no extra cost}: the pipeline already logs every candidate
and the stage at which it was filtered.
\textsc{SAM2INV} turns these logs into SFT records (program paired
with its certified annotation) and DPO preference pairs (certified
annotation as chosen; each rejected candidate as rejected, labelled
by its rejection stage).
This stage-stratified labelling provides a richer signal than binary
pass/fail, avoids reward hacking, and requires no human annotation
or online reward model.

\medskip
The remainder of this paper is organised as follows.
Section~\ref{sec:related} surveys related work.
Section~\ref{sec:overview} gives a high-level overview of the \textsc{SAM2INV} pipeline.
Section~\ref{sec:method} details each component.
Section~\ref{sec:loop-factory} describes \textsc{LoopFactory}.
Section~\ref{sec:experiments} presents experimental evaluation.
Section~\ref{sec:conclusion} concludes.

%% ============================================================
\section{Related Work}\label{sec:related}
%% ============================================================

\paragraph{Loop Invariant and Program Specification Generation.}
Classical invariant synthesis methods fall along a spectrum from fully static to fully dynamic.
Abstract interpretation~\cite{cousot1977abstract} computes fixed-point over-approximations in
abstract domains (intervals, octagons, polyhedra) and underpins industrial analysers, but is
restricted to the expressiveness of the chosen domain.
Template-based constraint solving~\cite{colon2003linear,sankaranarayanan2004non} fixes a
parametric invariant form and searches for satisfying coefficients via LP or SDP; the
approach is complete within the template family but cannot discover structure outside it.
The Houdini algorithm~\cite{flanagan2001houdini} starts from a large candidate set and
iteratively removes non-inductive conjuncts, converging to the strongest inductive
conjunction; we adopt this as the third stage of our filter pipeline.
Daikon~\cite{daikon} and DIG~\cite{nguyen2012using} mine likely invariants from execution
traces by instantiating templates over observed variable values or combining dynamic
analysis with symbolic execution; like us, they rely on traces as evidence, but provide no
formal inductiveness guarantee without a separate verifier pass.
Beyond invariants, there is a growing body of work on broader \emph{specification generation}:
inferring pre/postconditions~\cite{endres2024can}, separation-logic frame
conditions~\cite{berdine2005smallfoot}, and ACSL contracts from source code.
Our work targets loop invariant generation specifically, where the interaction between
inductiveness and the loop guard makes the problem particularly challenging.

\paragraph{LLM-Based Formal Verification Assistance.}
Recent work has explored using LLMs to assist formal verification.
Chakraborty and Lahiri~\cite{chakraborty2024towards} demonstrate that GPT-4 can generate
loop invariants for simple C programs given appropriate prompts and a verification
feedback loop.
Pei et al.~\cite{pei2023can} study code-trained LLMs producing Hoare-style annotations,
finding that iterative refinement with verifier feedback is essential.
Kamath et al.~\cite{kamath2023finding} propose a ``guess-and-check'' framework combining
LLM generation with bounded model checking.
In the Lean and Coq proof assistant ecosystem, LLMs have been used to suggest proof
tactics~\cite{han2021proof,jiang2022draft} and to translate informal mathematical
statements into formal proofs~\cite{wu2022autoformalization}.
Compared to these works, \textsc{SAM2INV} introduces a \emph{systematic three-stage
filter pipeline} that pre-screens candidates before formal verification,
dramatically reducing verifier load.
Moreover, by combining multiple LLM candidates through the pipeline, the resulting
invariant set is substantially stronger and more informative than any single-call
approach: the diversity of parallel generation and the merging step allow the system
to discover invariant conjuncts that no individual call would produce alone.

\paragraph{Automated Theorem Proving with Neural Models.}
Neural theorem proving has made rapid progress through tree-search over proof states
using LLMs as tactic
generators~\cite{han2021proof,lample2022hypertree,polu2022formal}.
AlphaCode~\cite{li2022competition} and similar systems apply large-scale sampling and
filtering for competitive programming, a paradigm closely related to ours.
A key difference is that deductive program verification with Frama-C/WP does not admit
a step-by-step proof search; instead, the verifier is an oracle that accepts or rejects
a complete annotation in one shot, making the feedback signal coarser and the filtering
pipeline more important.

\paragraph{Reinforcement Learning with Verifiable Rewards (RLVR).}
Reinforcement learning from human feedback~(RLHF)~\cite{ouyang2022training}
and direct preference optimisation~(DPO)~\cite{rafailov2023direct}
have become standard alignment techniques for LLMs, replacing scalar reward models
with pairwise preference data.
A natural extension replaces \emph{human} preferences with \emph{verifiable}
preferences derived from automated oracles, a paradigm known as reinforcement learning
with verifiable rewards~(RLVR).
DeepSeek-R1~\cite{deepseekr1} and related work~\cite{lambert2024tulu} demonstrate that
training on verifier-generated rewards for mathematical reasoning yields large capability
gains with no human annotation.
In code generation, execution-guided
methods~\cite{le2022coderl,gehring2024rlef} use test-case pass/fail signals to define
preferences, and process reward models~\cite{lightman2023lets} assign credit to
intermediate reasoning steps.
Our approach inherits the RLVR spirit but addresses a distinctive challenge:
in the loop invariant domain, binary verifier success is an especially sparse and
\emph{reward-hackable} signal, because a trivially weak invariant (e.g., \lstinline|true|)
may satisfy the verifier on programs without postconditions but is useless in practice.
We address this by generating \emph{structured} DPO preference pairs whose rejected
members are annotated by the \emph{stage} at which they fail (syntax, trace, or
formal inductiveness), providing a much richer training signal than a flat pass/fail
label.
The preference labels are ground-truth correct by construction---a verified invariant
truly satisfies Frama-C/WP's soundness conditions, and a rejected one truly does
not---making our setting particularly well suited to offline preference optimisation.

\paragraph{Benchmark Generation and Training Data for Program Verification.}
Existing invariant synthesis benchmarks---such as the NLA
(Non-Linear Arithmetic) suite from SV-COMP~\cite{svcomp}---contain
only a few dozen hand-crafted programs, making both systematic evaluation
and large-scale training data generation difficult.
\textsc{LoopFactory} addresses both needs: as the \emph{task input generator}
of \textsc{SAM2INV}, it generates large corpora of structured numeric loop programs
that the pipeline annotates and logs as SFT/DPO training records;
as a benchmark tool, it supports controllable ablation over program complexity
dimensions.
The design draws inspiration from grammar-based
fuzzing~\cite{godefroid2008grammar} and probabilistic program
synthesis~\cite{nori2015efficient}.

%% ============================================================
\section{System Overview}\label{sec:overview}
%% ============================================================

Figure~\ref{fig:pipeline} illustrates the end-to-end \textsc{SAM2INV} pipeline.
The system takes as input a C function containing a \texttt{while} loop,
a precondition (\texttt{requires}), and a postcondition (\texttt{assert}).
It outputs the same function annotated with ACSL loop invariants and a
\texttt{loop assigns} clause such that Frama-C/WP can fully verify the
postcondition, together with SFT and DPO training records for offline
policy fine-tuning.

\begin{figure}[t]
\centering
\fbox{\parbox{0.95\columnwidth}{%
\small
\begin{center}\textbf{\textsc{SAM2INV} Pipeline}\end{center}
\vspace{-0.5em}
\begin{enumerate}\setlength{\itemsep}{2pt}
\item[\ding{182}] \textbf{Smart Sampling} --- Execute the program with tiered inputs;
      collect execution traces recording variable values at every loop iteration.
\item[\ding{183}] \textbf{Parallel LLM Generation} --- Issue $N$ parallel LLM calls
      with diverse prompts and temperatures; parse and union all candidate invariants
      into a single pool.
\item[\ding{184}] \textbf{Stage~1 -- ACSL Syntax Filter} --- Reject candidates that
      violate ACSL syntax rules (unsupported quantifiers, forbidden operators, etc.).
\item[\ding{185}] \textbf{Stage~2 -- Execution-Trace Semantic Filter} --- Discard
      candidates falsified by any observed loop state without invoking the verifier.
\item[\ding{186}] \textbf{Stage~3 -- Houdini Formal Pruning} --- Iteratively remove
      non-inductive invariants via Frama-C/WP; the surviving set is \emph{certified}.
\item[\ding{187}] \textbf{Offline Data Collection} --- Record verified candidates as
      SFT data; pair them with rejected candidates as DPO preference pairs.
      Output annotated C code (or report failure).
\end{enumerate}
}}
\caption{The \textsc{SAM2INV} sampling-and-filtering pipeline.
Steps \ding{182}--\ding{183} form Phase~I (Sampling \& Generation);
steps \ding{184}--\ding{186} form Phase~II (Three-Stage Filtering);
step \ding{187} constitutes Phase~III (Certification \& Offline Data Collection).
Houdini pruning in step~\ding{186} is guaranteed to terminate.}
\label{fig:pipeline}
\end{figure}

The pipeline operates in three broad phases.

\paragraph{Phase~I: Sampling and Generation (Steps \ding{182}--\ding{183}).}
The target program is compiled and executed under a set of carefully chosen inputs
produced by the \emph{smart tiered sampler} (Section~\ref{sec:sampling}).
Execution traces---recording variable values at every loop iteration---are collected and
formatted into structured text that exposes the mathematical relationships maintained
by the loop.
Multiple LLM instances are then queried in parallel, each receiving a prompt assembled
from the source code, execution traces, and a system prompt encoding ACSL rules and an
assertion-driven synthesis strategy.
Prompt diversity is achieved by varying the template and the sampling temperature.
The union of all parsed responses forms the initial candidate pool.

\paragraph{Phase~II: Three-Stage Filtering (Steps \ding{184}--\ding{186}).}
The candidate pool is subjected to three progressively more expensive filters.
The \emph{ACSL syntax filter} checks each candidate against a set of structural
rules---no unsupported quantifiers, no forbidden operators, correct use of
\lstinline|\at| expressions---and discards violators immediately.
The \emph{execution-trace semantic filter} evaluates each surviving candidate against
every recorded loop state; any candidate falsified by at least one observed state is
discarded without invoking the verifier.
Finally, \emph{Houdini formal pruning} passes the pre-screened set to Frama-C/WP,
iteratively removing candidates whose verification conditions fail until the remaining
set is fully inductive.
Because at least one candidate is removed per Houdini iteration, termination is
guaranteed (Theorem~\ref{thm:termination}).
Every candidate that exits Phase~II is formally certified by Frama-C/WP.

\paragraph{Phase~III: Certification and Offline Data Collection (Step \ding{187}).}
The certified invariant set is written back into the source file as the verified output.
Simultaneously, the pipeline records two complementary training signals.
Candidates accepted through all three filter stages are recorded as \emph{chosen}
examples for supervised fine-tuning~(SFT) of an invariant-generation policy.
Candidates rejected at any earlier stage are paired with the corresponding chosen
examples to form \emph{direct preference optimisation}~(DPO) training pairs, providing
ground-truth preference labels derived entirely from formal verification---no human
annotation required.
This offline data collection adds negligible overhead: no extra LLM calls are made,
and the records are produced as a natural byproduct of pipeline execution.

\paragraph{Running Example.}
Consider the following program that computes the cube of~$n$:

\begin{lstlisting}
/*@ requires a>=n && n==0; */
int main1(int a, int n){
  int x, y, z;
  x=0; y=1; z=6;
  while(n <= a){
    n=n+1; x=x+y; y=y+z; z=z+6;
  }
  /*@ assert (n==a+1) && (y==3*n*n+3*n+1)
          && (x==n*n*n) && (z==6*n+6); */
}
\end{lstlisting}

The smart sampler executes this program with $a \in \{0,1,\ldots,10\}$
and collects traces such as:
\[
\small
\begin{array}{cccc}
\text{iter} & n & x & y \\
\hline
0 & 0 & 0 & 1 \\
1 & 1 & 1 & 7 \\
2 & 2 & 8 & 25 \\
\end{array}
\]
The parallel LLM generation step produces a pool of candidate invariants.
Ill-formed candidates (e.g., those using exponentiation \texttt{\^{}})
are eliminated by the syntax filter; candidates that evaluate to \texttt{false}
on any observed row are eliminated by the trace filter.
Houdini pruning then confirms the four inductive invariants
$x = n^3$, $y = 3n^2{+}3n{+}1$, $z = 6n{+}6$, and $n \le a{+}1$,
and Frama-C/WP verifies the postcondition in a single pass.
The verified invariants are stored as SFT data; every rejected candidate is
paired with this verified set to form a DPO training record.

%% ============================================================
\section{Method}\label{sec:method}
%% ============================================================

We now describe each component of \textsc{SAM2INV} in detail,
following the pipeline order introduced in Section~\ref{sec:overview}.

%% ------------------------------------------------------------
\subsection{Smart Dynamic Sampling}\label{sec:sampling}
%% ------------------------------------------------------------

The purpose of dynamic sampling is to collect execution traces that
expose the mathematical relationships maintained by the loop.
Naive random sampling often produces inputs that are either too
simple (all zeros) or too large (causing integer overflow or
time-outs).
We therefore adopt a \emph{tiered} sampling strategy that
prioritises simple, informative values while gradually introducing
diversity.

\begin{definition}[Value Tiers]
Given a parameter variable $v$ with domain $[l, u]$, we define
four value tiers:
\begin{align*}
\mathcal{T}_0 &= \{0, 1, -1\} \cap [l,u] && \text{(special values)} \\
\mathcal{T}_1 &= \{2,3,\ldots,10\} \cap [l,u] && \text{(small integers)} \\
\mathcal{T}_2 &\subseteq \{11,\ldots,50\} \cap [l,u] && \text{(medium, sampled)} \\
\mathcal{T}_3 &\subseteq \{51,\ldots,100\} \cap [l,u] && \text{(large, sampled)}
\end{align*}
\end{definition}

Sampling proceeds in phases:
\begin{enumerate}
\item \textbf{Phase~1:} Generate the Cartesian product over $\mathcal{T}_0$
      for all parameters.  These ``corner cases'' are ideal for fitting
      low-degree polynomials.
\item \textbf{Phase~2:} Mix $\mathcal{T}_0$ and $\mathcal{T}_1$ to add
      coverage of small positive integers.
\item \textbf{Phase~3:} Include all tiers, prioritising combinations with
      lower total tier index.
\item \textbf{Phase~4:} Fill remaining quota with biased random sampling
      (80\% probability of choosing values $\le 20$).
\end{enumerate}

Each input is executed, and the variable state at every loop iteration
is recorded.  To keep prompts within LLM context limits, we retain
only the first and last $k$ iterations per run (default $k{=}3$)
and group traces into at most $G$ groups (default $G{=}10$).

%% ------------------------------------------------------------
\subsection{Parallel LLM Generation}\label{sec:generation}
%% ------------------------------------------------------------

\paragraph{Prompt Design.}
Each LLM prompt is assembled from three components:
\begin{enumerate}
\item A \emph{system prompt} that defines the task (loop invariant
      synthesis for Frama-C/WP), lists allowed and forbidden ACSL
      constructs, and encodes the assertion-driven synthesis strategy
      (decompose postcondition conjuncts, add boundary invariants,
      preserve unmodified parameters).
\item The \emph{source code} of the target function.
\item \emph{Execution traces} formatted as structured variable--value
      tables (Section~\ref{sec:sampling}).
\end{enumerate}

\paragraph{Assertion-Driven Strategy.}
A key insight is that the postcondition assertion directly suggests
invariant candidates:
\begin{enumerate}
\item Each conjunct of the assertion (split on \texttt{\&\&}) is a
      candidate invariant.
\item For a loop guard of the form \lstinline|i < n|, the weakened
      form \lstinline|i <= n| is an inductive boundary invariant.
\item Function parameters not modified in the loop body are preserved:
      \lstinline|v == \at(v, Pre)|.
\end{enumerate}

\paragraph{Parallel Diverse Generation and Union Assembly.}
To maximise the probability of finding the correct invariant set, we
issue $N$ parallel LLM calls (default $N{=}5$) using a thread pool,
with temperatures $\tau_j$ sampled from $\{0.8, 1.0, 1.1, 1.2\}$
to encourage diverse outputs.
Each call $j$ produces a set of ACSL expression strings $C_j$.
These sets are \emph{unioned}, not voted upon:
\[
C_{\text{pool}} = \bigcup_{j=1}^{N} C_j \quad
(\text{whitespace-normalised deduplication}).
\]
The union $C_{\text{pool}}$ is then passed as a single flat set to
the three-stage filter (Section~\ref{sec:filtering}).

The union-then-filter design is motivated by a structural property
of loop invariants: the correct annotation is a \emph{conjunction}
of multiple clauses, and different clauses tend to be discovered by
different calls.
After Houdini pruning, the certified output $C^*$ is assembled from
whichever clauses in $C_{\text{pool}}$ survive---it is a post-hoc
collective result, not a selection from any individual response.
This \emph{non-attributability} property has direct consequences for
training signal design; see Section~\ref{sec:offline} and
Appendix~\ref{app:nonattrib} for the formal analysis.

%% ------------------------------------------------------------
\subsection{Three-Stage Filter Pipeline}\label{sec:filtering}
%% ------------------------------------------------------------

The three-stage filter pipeline is the core engine of \textsc{SAM2INV}.
It improves invariant generation in two independent ways.

First, \emph{union assembly strengthens the invariant.}
By pooling candidates from $N$ parallel calls, the union contains
conjuncts that no individual call would produce on its own.
After Houdini pruning, the certified $C^*$ is the strongest
inductive conjunction the union supports---strictly stronger, in
general, than the output of any single call
(Section~\ref{sec:generation}, Appendix~\ref{app:nonattrib}).

Second, \emph{the lightweight filters ensure Houdini can work
correctly and efficiently.}
The syntax filter removes expressions that are syntactically
malformed ACSL---parse errors, forbidden constructs, unbalanced
brackets---which would cause Frama-C to fail entirely rather
than returning a per-invariant result.
The trace filter removes candidates that are \emph{semantically
contradicted} by observed execution states: any candidate falsified
by at least one recorded loop state cannot be inductive and need
never be sent to Frama-C.
Together, these two stages deliver a pre-screened candidate set
to Houdini, eliminating the risk of Frama-C parse failures and
substantially reducing the number of formal verification calls.

Crucially, both improvements are \emph{intrinsic to the pipeline
itself}---they hold for any base LLM, before any fine-tuning.
The offline training data collection (Section~\ref{sec:offline}) is
a separate improvement layer built on top of this already-strong baseline.

The design principle across stages is to discard as many
incorrect candidates as possible \emph{before} invoking the
more expensive stage that follows.

\paragraph{Stage~1: ACSL Syntax Filter.}
Each candidate is checked against a set of structural ACSL rules
before any execution or verification is attempted:
\begin{itemize}
\item No quantifiers (\lstinline|\forall|, \lstinline|\exists|),
      which are unsupported by Frama-C/WP for loop invariants.
\item No custom definitions (\texttt{predicate}, \texttt{logic},
      \texttt{lemma}).
\item No ternary operator (\texttt{? :}).
\item \lstinline|\at(v, Pre)| permitted only on function parameters,
      not on local variables.
\item No exponentiation (\texttt{\^{}} is bitwise XOR in C/ACSL;
      repeated multiplication must be used instead).
\end{itemize}
Candidates failing any rule are discarded immediately.
This filter runs in microseconds per candidate and eliminates a
substantial fraction of the pool before any program execution is
needed.

\paragraph{Stage~2: Execution-Trace Semantic Filter.}
Each surviving candidate is evaluated as a boolean expression over
every recorded loop state in the collected execution traces.
A candidate that evaluates to \texttt{false} on any observed state
is \emph{inconsistent} with the program's actual behaviour and is
discarded.
While trace evaluation does not guarantee inductiveness---the
traces are finite and cannot cover all reachable states---it
eliminates candidates with grossly incorrect arithmetic relationships
cheaply, without invoking Frama-C.
In practice, this stage reduces the candidate pool by a further
significant fraction before the formal verifier is engaged.

\paragraph{Stage~3: Houdini Formal Pruning.}
After the two lightweight filters, the surviving candidates are
submitted to Frama-C/WP for formal verification.
We apply the Houdini algorithm~\cite{flanagan2001houdini}:

\begin{figure}[t]
\centering
\fbox{\parbox{0.92\columnwidth}{\small
\textbf{Algorithm: Houdini Pruning}\\[2pt]
\textbf{Input:} Candidate set $C$, verifier $\mathcal{V}$\\
\textbf{Output:} Maximal inductive subset $C^* \subseteq C$\\[2pt]
1: $C^* \leftarrow C$\\
2: \textbf{repeat}\\
3: \quad $R \leftarrow \mathcal{V}(C^*)$ \hfill\textit{// verify each invariant}\\
4: \quad $F \leftarrow \{c \in C^* \mid R(c) = \textit{fail}\}$\\
5: \quad $C^* \leftarrow C^* \setminus F$\\
6: \textbf{until} $F = \emptyset$ \textbf{or} $C^* = \emptyset$\\
7: \textbf{return} $C^*$
}}
\caption{Houdini-style pruning algorithm used in Stage~3 of the filter pipeline.}
\label{alg:houdini}
\end{figure}

\begin{theorem}[Termination]\label{thm:termination}
Algorithm~\ref{alg:houdini} terminates in at most $|C|$ iterations,
since each iteration removes at least one candidate.
\end{theorem}

The output $C^*$ is the maximal subset of the pre-screened candidates
that is simultaneously inductive and collectively sufficient to
discharge the Frama-C/WP verification conditions.
Every invariant in $C^*$ is formally certified.
If $C^*$ is non-empty and sufficient to prove the postcondition,
the pipeline reports success.
If the pruned set is non-empty but insufficient, the system may
optionally enter a brief iterative repair loop (at most $K$ iterations;
default $K{=}3$) in which verification error messages are fed back
to the LLM for targeted strengthening; this repair step is not the
primary mechanism and is not counted as part of the three-stage filter.

%% ------------------------------------------------------------
\subsection{Offline Policy Exploration}\label{sec:offline}
%% ------------------------------------------------------------

A distinctive property of the \textsc{SAM2INV} pipeline is that it
produces not only a certified invariant annotation but also
high-quality training data for an invariant-generation policy---as
a byproduct of every pipeline run.

\paragraph{Why not GRPO.}
A natural baseline is group-relative policy
optimisation~(GRPO)~\cite{deepseekr1}, which scores each individual
completion via binary verifier feedback.
Two structural properties make this unsuitable here.
First, \emph{non-attributability} (Section~\ref{sec:generation},
Appendix~\ref{app:nonattrib}): the correct annotation $C^*$ is
assembled post-hoc from the union of $N$ calls, so a call that
contributes only one essential conjunct will fail binary verification
on its own yet be indispensable to the final answer---penalising it
degrades the diversity that makes the union effective.
Second, \emph{binary reward hackability}: trivially weak annotations
such as \lstinline|loop invariant true;| can formally pass the
verifier on programs with loose postconditions, making binary
pass/fail an unreliable quality signal.

\paragraph{DPO data collection.}
Both problems are avoided by \emph{offline} preference optimisation.
The pipeline logs every candidate and the stage at which it was
filtered; this log is converted into DPO training records at no
extra cost.
Let $C^*$ be the certified set and $P^+$ the annotated program
obtained by inserting $C^*$.
For each rejected candidate $r$ eliminated at stage
$k \in \{1,2,3\}$, we form the record $(P,\, P^+,\, P^-_r,\, k)$
where $P^-_r$ inserts $\{r\}$ as the sole annotation.
The preference label is ground-truth correct by construction.
Rejected candidates are weighted by stage: syntax failures ($k{=}1$)
receive lower weight than inductiveness failures ($k{=}3$), since
the latter represent the hardest and most informative contrasts.
The SFT record $(P,\, P^+)$ is generated alongside each successful
pipeline run.
Formal definitions and the stage-stratified loss are given in
Appendix~\ref{app:dpo}.

\paragraph{Policy improvement cycle.}
Fine-tuning on the accumulated SFT and DPO data produces a model
that is deployed in place of the base LLM in Step~\ding{183}.
The filter pipeline certifies every output regardless of model
quality---a weaker model yields a smaller $C^*$ but never an
uncertified one---so the system remains sound throughout the
improvement cycle.

%% ============================================================
\section{Probabilistic Loop Synthesis}\label{sec:loop-factory}
%% ============================================================

Existing benchmarks for loop invariant synthesis (e.g., the NLA suite)
contain only a few dozen programs, making it difficult to evaluate
invariant synthesis tools at scale or to study the effect of program
complexity on synthesis success.
To address this, we design \textsc{LoopFactory}, a probabilistic
domain-specific language (DSL) for generating structured numeric
loop programs with controllable complexity.

%% ------------------------------------------------------------
\subsection{Hyperparameters}
%% ------------------------------------------------------------

The generator is parameterised by:
\[
\theta = (m,\; p,\; n,\; k,\; D_{\max},\;
          \pi_{\text{op}},\;
          \pi_{\text{cmp}},\;
          \pi_{\text{const}},\;
          \pi_{\text{self}},\;
          \pi_{\text{nest}})
\]
where $m$ is the maximum number of variables, $p$ the number of
(immutable) parameter variables, $n$ the number of top-level loops,
$k$ the maximum assignments per loop body, $D_{\max}$ the maximum
nesting depth, and the remaining symbols are probability distributions
governing operator choice, constant injection, self-update assignment,
and loop nesting (Table~\ref{tab:hyperparams}).

\begin{table}[t]
\caption{Hyperparameters of the \textsc{LoopFactory} DSL.}
\label{tab:hyperparams}
\centering\small
\begin{tabular}{@{}ll@{}}
\toprule
Symbol & Meaning \\
\midrule
$m$ & Max variables \\
$p$ & Parameter (immutable) variables \\
$n$ & Top-level loops \\
$k$ & Max assignments per loop body \\
$D_{\max}$ & Max nesting depth \\
$\pi_{\text{op}}$ & Distribution over $\{+,-,\times,/,\bmod\}$ \\
$\pi_{\text{cmp}}$ & Distribution over $\{<,\le,>,\ge,=,\ne\}$ \\
$\pi_{\text{const}}$ & Prob.\ of sampling a constant operand \\
$\pi_{\text{self}}$ & Prob.\ of self-update ($v_i := v_i \;\texttt{op}\; x$) \\
$\pi_{\text{nest}}$ & Prob.\ of generating a nested sub-loop \\
\bottomrule
\end{tabular}
\end{table}

%% ------------------------------------------------------------
\subsection{DSL Syntax}
%% ------------------------------------------------------------

\paragraph{Expressions.}
Arithmetic expressions are of the form:
\[
e \;::=\; v \;\mid\; c \;\mid\; v \;\texttt{op}\; x
\]
where $v \in \mathcal{V}$ is a variable, $c \in \mathbb{Z}$ is an
integer constant, $x \in \mathcal{V} \cup \mathbb{Z}$, and
$\texttt{op} \in \{+,-,\times,/,\bmod\}$.
If $\texttt{op} = \bmod$, the right operand must be a positive
integer constant.

\paragraph{Boolean Guards.}
Loop guards are comparisons $e_1 \;\texttt{cmp}\; e_2$ with
$\texttt{cmp} \in \{<,\le,>,\ge,=,\ne\}$.

\paragraph{Assignments.}
An assignment $v := e$ requires $v \in \mathcal{L}$, where
$\mathcal{L} = \mathcal{V} \setminus \mathcal{P}$ is the set
of writable (non-parameter) variables.

%% ------------------------------------------------------------
\subsection{Program Structure}
%% ------------------------------------------------------------

A generated program has the form:
\[
\textbf{Prog} \;::=\; \textbf{Init} \;;\; \textbf{LoopForest}
\]

The \textbf{Init} block assigns every writable variable exactly once,
using expressions that depend only on parameter variables,
ensuring a well-defined initial state.

The \textbf{LoopForest} is a sequence of loop trees
$\mathcal{F} = [T_1, \ldots, T_n]$, where each tree node
\[
T = \langle\, b,\; S,\; \mathcal{C} \,\rangle
\]
consists of a guard~$b$, an assignment list~$S$ with $|S| \le k$,
and a (possibly empty) list of child loops~$\mathcal{C}$.
Nesting depth is bounded by $D_{\max}$.

%% ------------------------------------------------------------
\subsection{Semantic Loop Templates}
%% ------------------------------------------------------------

To ensure that generated loops exhibit mathematically interesting
behaviour (rather than trivial or divergent patterns), the DSL
includes twelve \emph{semantic templates} that encode common
loop idioms:

\begin{enumerate}\setlength{\itemsep}{1pt}
\item Cubic growth recursion ($x, y, z$ coupled updates)
\item Geometric recursion ($x = x \cdot z + 1$)
\item Cumulative sum / dot product
\item Power-sum accumulation
\item Quotient--remainder counting
\item Binary multiplication style
\item Linear dual-variable shifting
\item Linear decrement to zero
\item Modulo bucket counting
\item Euclidean-style subtraction
\item Nested triangular sums
\item Nested affine updates
\end{enumerate}

Each template defines a fixed update pattern but allows the
generator to sample concrete operators and constants from the
probabilistic model, yielding a family of programs per template.

%% ------------------------------------------------------------
\subsection{Generative Distribution}
%% ------------------------------------------------------------

The probability of a complete program factorises as:
\begin{equation}\label{eq:joint}
\Pr(\textbf{Prog}) =
  \prod_{v \in \mathcal{L}} \Pr(v := e_v)
  \;\cdot\;
  \prod_{T \in \mathcal{F}} \Pr(T)
\end{equation}
where, recursively,
\[
\Pr(T) = \Pr(b) \cdot \prod_{s \in S} \Pr(s) \cdot
          \prod_{T' \in \mathcal{C}} \Pr(T').
\]

This factorisation ensures that the generator defines a proper
probability distribution over the space of well-formed loop programs,
enabling statistical analysis of benchmark difficulty and ablation
over individual hyperparameters.

\paragraph{Expected Properties.}
Under the Bernoulli nesting model, the expected nesting depth is:
\[
\mathbb{E}[\text{depth}] =
  \sum_{d=0}^{D_{\max}} d \cdot (1 - \pi_{\text{nest}})
  \,\pi_{\text{nest}}^d
\]
and the expected number of loop nodes is
$\sum_{d=0}^{D_{\max}} \pi_{\text{nest}}^d$.
These closed-form expressions allow practitioners to tune
$\pi_{\text{nest}}$ and $D_{\max}$ to obtain a desired distribution
of benchmark difficulty.

%% ============================================================
\section{Experiments}\label{sec:experiments}
%% ============================================================

%% TODO: Fill in experimental results.
%% Suggested structure:
%%
%% \subsection{Experimental Setup}
%%   - Benchmarks: NLA suite, LoopFactory-generated programs
%%   - LLM models: GPT-4o, GPT-5-nano, DeepSeek-V3.1
%%   - Baselines: Daikon, DIG, direct GPT-4 prompting
%%   - Metrics: success rate, #LLM queries, #Frama-C calls, token cost
%%   - Hardware / timeout configuration
%%
%% \subsection{RQ1: Overall Effectiveness}
%%   - Table: per-benchmark success rate
%%   - Comparison with baselines
%%
%% \subsection{RQ2: Ablation Study}
%%   - Effect of smart sampling vs. random sampling
%%   - Effect of parallel diverse generation (N=1 vs N=5 vs N=20)
%%   - Effect of Houdini pruning
%%   - Effect of generation mode (code_only vs fit_only vs hybrid)
%%
%% \subsection{RQ3: Scalability on Synthetic Benchmarks}
%%   - Success rate vs. program complexity (nesting depth, variable count)
%%   - Token cost analysis
%%
%% \subsection{RQ4: Vector Cache Effectiveness}
%%   - Cache hit rate over time
%%   - Reduction in LLM queries for similar programs
%%
%% \subsection{Threats to Validity}

%% ============================================================
\section{Conclusion}\label{sec:conclusion}
%% ============================================================

We have presented \textsc{SAM2INV}, an end-to-end system for automatic
loop invariant synthesis that tightly integrates large language models
with dynamic sampling and formal verification.
The three pillars of our approach---smart tiered sampling, parallel
diverse generation, and Houdini-style iterative pruning---address
complementary weaknesses of prior LLM-based methods:
sampling grounds the LLM in concrete program behaviour,
diversity increases the probability of generating the correct invariant,
and Houdini pruning provides monotonic progress guarantees within the
verification loop.

Our probabilistic loop synthesis DSL, \textsc{LoopFactory}, enables
controlled large-scale evaluation beyond the small existing benchmarks,
and its factorised generative distribution allows principled ablation
over program complexity dimensions.

%% TODO: Summarise experimental findings once available.

Several directions remain for future work.
First, extending the system to handle pointer-manipulating programs
and separation-logic invariants (via Coq/VST integration, which our
codebase already partially supports) would broaden the applicability
of the approach.
Second, replacing the prompt-based LLM interaction with fine-tuned
models trained on verified invariant datasets may improve both
accuracy and efficiency.
Third, integrating symbolic techniques---such as abstract
interpretation for computing initial bounds or interpolation for
strengthening failing invariants---could provide a hybrid
symbolic--neural architecture with stronger guarantees.

%% ============================================================
\appendix
%% ============================================================

%% ============================================================
\section{\textsc{LoopFactory} DSL: Sampling Procedure and Concrete Example}
\label{app:loop-factory}
%% ============================================================

\subsection{Full Generation Procedure}

Algorithm~\ref{alg:loop-factory} shows the complete \textsc{LoopFactory}
generation loop.
Starting from a drawn hyperparameter vector~$\theta$ (or a fixed
configuration), the factory (i)~selects parameters from a candidate pool,
(ii)~samples loop count and, for each loop, the NLA vs.\ linear
family classification, (iii)~allocates fresh single-letter variable names
for each loop's counter and limit, (iv)~samples loop control modes
(increment, decrement, multiplicative, distance-to-limit, etc.),
(v)~fills each loop body with self-update and peer-update assignments
drawn from $\pi_{\text{op}}$, and (vi)~optionally inserts semantic-core
assignments from the twelve hardcoded idiom templates.
The rendered program is then wrapped in a function signature with a
Frama-C \texttt{requires} clause and passed downstream for execution and
invariant generation.

\begin{figure}[htb]
\centering
\fbox{\parbox{0.92\columnwidth}{\small
\textbf{Algorithm: \textsc{LoopFactory} Generation}\\[2pt]
\textbf{Input:} Hyperparameters $\theta$, seed $s$\\
\textbf{Output:} A well-typed C function\\[2pt]
1: $\mathit{rng} \leftarrow \text{Random}(s)$\\
2: $\mathcal{P} \leftarrow \text{sample}(\text{candidates}, p,\; \mathit{rng})$
   \hfill\textit{// e.g.\ \{a,n\}}\\
3: $n_{\ell} \leftarrow \text{Uniform}[\min\_\text{fuel},\; \text{while\_fuel}]$
   \hfill\textit{// loop count}\\
4: Init block: for $v \in \mathcal{L}$, sample $e_v$ over $\mathcal{P}$ only\\
5: \textbf{for} $i = 1,\ldots,n_{\ell}$ \textbf{do}\\
6: \quad Allocate counter $\mathit{ctr}$, limit $\mathit{lim}$ via \texttt{NameAllocator}\\
7: \quad $\mathit{nla} \leftarrow \text{Bernoulli}(p_{\text{nonlinear}})$\\
8: \quad $(\text{inits},\; \text{guard},\; \text{step}) \leftarrow
   \texttt{SampleLoopControl}(\mathit{ctr},\mathit{lim},\mathit{nla})$\\
9: \quad Body $\leftarrow$ [step]\\
10:\quad \textbf{repeat} up to $k-1$ times:\\
11:\quad\quad Sample assign $\leftarrow \texttt{SemanticAssign}(\mathit{nla})$\\
12:\quad\quad Body $\leftarrow$ Body $\cup$ \{assign\}\\
13:\quad With prob.\ $p_{\text{semantic\_core}}$: inject one template idiom\\
14:\quad With prob.\ $q_{\text{nest}}$: recurse to depth $d{+}1 \le D_{\max}$\\
15:\quad Append \texttt{WhileLoop(guard, Body)} to forest\\
16:\textbf{end for}\\
17: \textbf{return} \texttt{Program}($\mathcal{P}$, inits, forest).render()
}}
\caption{\textsc{LoopFactory} generation algorithm.}
\label{alg:loop-factory}
\end{figure}

\subsection{Loop-Control Mode Sampling}

The function \texttt{SampleLoopControl} draws a loop-control mode from
seven shapes (Table~\ref{tab:loop-control}).
Weights differ between the NLA and linear families: multiplicative and
distance-to-limit modes are disabled for linear programs.

\begin{table}[htb]
\caption{Loop-control modes and their effect on counter, guard, and step.}
\label{tab:loop-control}
\centering\small
\begin{tabular}{@{}lll@{}}
\toprule
Mode & Guard shape & Step \\
\midrule
\texttt{inc1}      & $\mathit{ctr} < \mathit{lim}$ & $\mathit{ctr} \mathrel{+}= 1$ \\
\texttt{dec1}      & $\mathit{ctr} > 0$            & $\mathit{ctr} \mathrel{-}= 1$ \\
\texttt{inc\_step} & $\mathit{ctr} < \mathit{lim}$ & $\mathit{ctr} \mathrel{+}= d,\; d\in[2,5]$ \\
\texttt{dec\_step} & $\mathit{ctr} > d{-}1$        & $\mathit{ctr} \mathrel{-}= d$ \\
\texttt{mul\_up}   & $\mathit{ctr} < \mathit{lim}$ & $\mathit{ctr} \mathrel{*}= m,\; m\in\{2,3\}$ \\
\texttt{div\_down} & $\mathit{ctr} > 0$            & $\mathit{ctr} \mathrel{/}= 2$ \\
\texttt{dist\_to\_limit} & $\mathit{ctr} > \mathit{lim}$ & $\mathit{ctr} \mathrel{-}= d$ \\
\bottomrule
\end{tabular}
\end{table}

\subsection{Concrete Generated Example}

The following function was generated by \textsc{LoopFactory} with
$\theta = (m{=}5,\; p{=}2,\; \text{while\_fuel}{=}2,\;
k{=}4,\; D_{\max}{=}1,\; p_{\text{nonlinear}}{=}0.55)$:

\begin{lstlisting}
/*@ requires a >= 0 && n >= 1; */
int f(int a, int n){
  int c, s, x;
  c=0; s=a; x=1;

  while (c < n) {      // inc1: ctr=c, lim=n
    c = c+1;
    s = s+c;           // semantic: cumulative sum
    x = x*c;           // NLA: factorial-style
  }

  while (x > 0) {      // div_down: ctr=x
    x = x/2;
  }
  /*@ assert s == a + n*(n+1)/2; */
}
\end{lstlisting}

This single program exercises: (i)~an NLA loop with invariants
$s = a + c(c+1)/2$ and $x = c!$; (ii)~a geometric-decay loop
requiring $x \ge 0$; (iii)~a postcondition that tests only the
sum invariant, so the \textsc{SAM2INV} pipeline must discover the
correct invariant by combining candidates from multiple LLM calls.

%% ============================================================
\section{Detailed Invariant Generation Algorithm}
\label{app:algorithm}
%% ============================================================

\subsection{Data Structures}

The pipeline maintains the following state for each outer loop index $i$:

\begin{itemize}
\item $\mathbf{record}_i$: the structured descriptor extracted from
  static analysis, containing the loop guard, assignments (transition
  relation), initial variable values, invariable (unmodified) parameters,
  and function parameter names.
\item $\mathcal{T}_i$: the execution trace set, a list of
  variable-state snapshots collected by the smart sampler.
\item $C_i$: the live candidate pool (a set of ACSL expression strings).
\item $C^*_i$: the certified invariant set (output).
\end{itemize}

\subsection{Complete Per-Loop Algorithm}

\begin{figure}[htb]
\centering
\fbox{\parbox{0.92\columnwidth}{\small
\textbf{Algorithm: InvariantGenerator per loop $i$}\\[2pt]
\textbf{Input:} Source code $P$, record $\mathbf{r}$, traces $\mathcal{T}$,
               LLM $\mathcal{M}$, verifier $\mathcal{V}$\\
\textbf{Output:} Annotated code $P^*$ or $\bot$\\[2pt]
\textit{Phase 1 — Generation:}\\
1: $P_{\text{tmpl}} \leftarrow \texttt{InsertTemplate}(P, i)$
   \hfill\textit{// PLACE\_HOLDER annotations}\\
2: $(\sigma, \ell) \leftarrow \texttt{PreparePrompt}(\mathbf{r}, \mathcal{T})$
   \hfill\textit{// (system, user) prompts}\\
3: \textbf{for} $j = 1,\ldots,N$ \textbf{in parallel} \textbf{do}\\
4: \quad $R_j \leftarrow \mathcal{M}(P_{\text{tmpl}}, \sigma, \ell,\;\tau_j)$
   \hfill\textit{// diverse temperature}\\
5: \quad $C_j \leftarrow \texttt{ExtractInvariants}(R_j)$\\
6: \textbf{end for}\\
7: $C \leftarrow \bigcup_j C_j$ \hfill\textit{// union of all candidates}\\[4pt]
\textit{Phase 2a — Heuristic ACSL Syntax Gate:}\\
8: $C \leftarrow \{c \in C \mid \texttt{SyntaxFilter}(c,\mathbf{r})= \text{pass}\}$\\[4pt]
\textit{Phase 2b — Execution-Trace Semantic Gate:}\\
9: $C \leftarrow \{c \in C \mid \forall \mathbf{s} \in \mathcal{T}:\;
   \texttt{eval}(c, \mathbf{s}) = \text{true}\}$\\[4pt]
\textit{Phase 2c — Conflict Detection:}\\
10: $C \leftarrow \texttt{RemoveConflicts}(C)$\\[4pt]
\textit{Phase 2d — Merge:}\\
11: $C_{\text{merged}} \leftarrow \texttt{Dedup}(C)$
    \hfill\textit{// whitespace-normalised}\\[4pt]
\textit{Phase 3 — Houdini Formal Pruning:}\\
12: $C^* \leftarrow \texttt{Houdini}(C_{\text{merged}}, \mathcal{V})$
    \hfill\textit{// see Algorithm~\ref{alg:houdini}}\\
13: \textbf{if} $C^* = \emptyset$ \textbf{then return} $\bot$\\[4pt]
\textit{Phase 4 — Postcondition Check \& Optional Repair:}\\
14: \textbf{if} $\mathcal{V}$ reports \texttt{assert} satisfied \textbf{then}\\
15: \quad \textbf{return} $\texttt{Annotate}(P, C^*)$\\
16: \textbf{else} run $\le K$ repair iterations (feed error to $\mathcal{M}$)\\
17: \textbf{return} $\texttt{Annotate}(P, C^*)$ or $\bot$
}}
\caption{Full per-loop invariant generation algorithm.}
\label{alg:full}
\end{figure}

\subsection{Role of Static Analysis}

Before Phase~1, the sampler's \texttt{LoopAnalysis} module performs a
\emph{lightweight static pre-pass} on the input C file to populate
$\mathbf{record}_i$.
This is \emph{not} symbolic execution in the strict sense; it is a
source-level analysis that:
\begin{enumerate}
\item Identifies the loop guard (transition relation) by reading
  the \texttt{condition} field recorded in the JSON analysis file
  produced by the Frama-C preprocessing step.
\item Extracts the set of variables that are never assigned inside
  the loop body (\emph{unchanged variables}), which become candidates
  for $\texttt{\textbackslash at}(v, \text{Pre})$ invariants.
\item Infers conservative integer bounds $[\ell_v, u_v]$ for each
  parameter via \texttt{LoopBoundAnalyzer}, which are used to
  constrain the smart sampler's input domain.
\item Builds the \textbf{record} structure that encodes loop content,
  transition expression, pre-condition, and the names of function
  parameters (needed to gate $\texttt{\textbackslash at}$ usage in the
  syntax filter).
\end{enumerate}
The Frama-C/WP oracle is \emph{only} invoked during the Houdini
pruning stage (Phase~3) and never during candidate generation or
lightweight filtering.

\subsection{Non-Attributability of the Certified Output}
\label{app:nonattrib}

We formalise the key structural property stated informally in
Section~\ref{sec:generation}.

\begin{definition}[Attribution]
Given $N$ candidate sets $C_1,\ldots,C_N$ and a certified output
$C^*$, we say $C^*$ is \emph{attributable to call $j$} if
$C^* \subseteq C_j$.
\end{definition}

\begin{proposition}[Non-attributability is typical]
\label{prop:nonattrib}
Suppose the correct annotation for a loop requires a conjunction of
$m \ge 2$ invariant clauses $c_1,\ldots,c_m$.
If each clause $c_i$ is discovered independently by each of the
$N$ calls with probability $q \in (0,1)$, then the probability that
\emph{all} clauses appear in any single call $C_j$ is $q^m$,
while the probability that all clauses appear in the union
$C_{\text{pool}} = \bigcup_j C_j$ is $1 - (1-q)^{mN}$.
For $q = 0.6$, $m = 4$, $N = 5$: single-call coverage $= 0.6^4 \approx 0.13$;
union coverage $= 1 - 0.4^{20} \approx 1.00$.
\end{proposition}

\begin{proof}
Follows directly from independence and the complement rule for
union of events.
\end{proof}

Proposition~\ref{prop:nonattrib} shows that the gap between
single-call and union coverage grows exponentially with the number
of required conjuncts $m$.
In practice, real loop invariants for arithmetic programs require
$m \in [2, 6]$ conjuncts (boundary invariants, postcondition
conjuncts, auxiliary equalities), making single-call attribution
unlikely in typical cases.

\begin{corollary}
Under the conditions of Proposition~\ref{prop:nonattrib} with
$m \ge 2$ and $q < 1$, the expected fraction of per-call outputs
$C_j$ for which $C^* \subseteq C_j$ is $q^m < 1$.
Assigning a positive reward to calls satisfying $C^* \subseteq C_j$
and a negative reward to others would on average penalise
$(1 - q^m) > 0$ fraction of calls that may have contributed
essential clauses to the union.
\end{corollary}

This corollary shows that any \emph{per-call} binary scoring scheme
--- including GRPO --- systematically mislabels calls that contribute
essential clauses to the final answer, degrading the training signal
for exactly the behaviours (diverse conjunct coverage) that make the
union effective.

%% ============================================================
\section{DPO Data Construction: Formal Specification}
\label{app:dpo}
%% ============================================================

\subsection{Chosen Response Construction}

Given a program $P$ and the certified invariant set $C^*$ for loop
$i$, the chosen annotated program is:
\[
P^+ = \texttt{Annotate}(P,\; i,\; C^*)
\]
where \texttt{Annotate} inserts the ACSL block
\begin{center}
\texttt{/*@ loop invariant $c_1$; \ldots loop invariant $c_k$;\\
\phantom{/*@} loop assigns \textit{vars}; */}
\end{center}
immediately before the \texttt{while} statement of loop $i$.
The $\texttt{loop assigns}$ clause is derived from the set of
variables assigned anywhere in the loop body.
$P^+$ is guaranteed to satisfy Frama-C/WP because it survived
Stage~3 (Houdini pruning); no additional verification is needed
for the training record.

\subsection{Rejected Response Construction and Stage Labelling}

For each $r \in C_{\text{pool}} \setminus C^*$, we construct:
\[
P^-_r = \texttt{Annotate}(P,\; i,\; \{r\})
\]
and record the stage $k(r) \in \{1, 2, 3\}$ at which $r$ was
eliminated:
\begin{itemize}
\item $k(r) = 1$: $r$ failed the ACSL syntax gate
  (Appendix~\ref{app:syntax-gate}).
\item $k(r) = 2$: $r$ passed syntax but was falsified by some
  trace state $\mathbf{s} \in \mathcal{T}$, i.e.\
  $\texttt{eval}(r, \mathbf{s}) = \text{false}$.
\item $k(r) = 3$: $r$ passed syntax and trace filters but was
  removed by Houdini because Frama-C/WP could not verify
  its establishment or preservation condition.
\end{itemize}
The full DPO training record for program $P$, loop $i$, and
rejected expression $r$ is the tuple:
\[
\mathcal{D}(P, i, r) = \bigl(P,\; P^+,\; P^-_r,\; k(r)\bigr).
\]

\subsection{Stage-Stratified Loss Weighting}

Standard DPO minimises the loss:
\[
\mathcal{L}_{\text{DPO}} =
  -\mathbb{E}\!\left[\log \sigma\!\left(
    \beta \log \frac{\pi_\theta(P^+ \mid P)}{\pi_{\text{ref}}(P^+ \mid P)}
    - \beta \log \frac{\pi_\theta(P^-_r \mid P)}{\pi_{\text{ref}}(P^-_r \mid P)}
  \right)\right].
\]
We extend this with stage-dependent weights $w_k$:
\[
\mathcal{L}_{\text{S-DPO}} =
  -\mathbb{E}\!\left[w_{k(r)} \cdot \log \sigma\!\left(
    \beta \log \frac{\pi_\theta(P^+ \mid P)}{\pi_{\text{ref}}(P^+ \mid P)}
    - \beta \log \frac{\pi_\theta(P^-_r \mid P)}{\pi_{\text{ref}}(P^-_r \mid P)}
  \right)\right].
\]
The rationale for $w_1 \le w_2 \le w_3$ is that Stage-3 rejections
are the hardest preference distinctions: the rejected expression is
consistent with all observed program states and was eliminated only
by a formal inductiveness check, making the margin between $P^+$ and
$P^-_r$ the smallest and the training signal the most informative.
Concretely, we use $w_1 = 0.5,\; w_2 = 1.0,\; w_3 = 2.0$ as
default weights, reflecting the increasing difficulty of the
corresponding rejection judgements.

\subsection{SFT Records}

Alongside DPO pairs, each successful pipeline run produces an SFT
record $(P,\; P^+)$.
SFT pre-training warms up the model towards the format and style
of ACSL annotations before DPO alignment, following the standard
two-phase fine-tuning recipe~\cite{ouyang2022training}.
Because $P^+$ is formally verified, SFT on these records teaches
the model what a \emph{correct and informative} annotation looks
like, providing a positive learning signal that complements the
contrastive DPO signal.

%% ============================================================
\section{Prompts Used in Each Stage}
\label{app:prompts}
%% ============================================================

\subsection{System Prompt}

The system prompt is loaded verbatim from \texttt{prompts/system\_prompt.txt}
and prepended to every LLM call. Its full content is reproduced below.

\begin{lstlisting}[style=coqstyle, basicstyle=\ttfamily\scriptsize]
You are a formal verification expert specializing in loop
invariant synthesis for Frama-C with the WP plugin.

## TASK
Given a C function with a loop and a postcondition
(/*@ assert ... */), generate ACSL loop annotations
(invariants, assigns) that enable Frama-C/WP to verify
the assertion.

## CORE STRATEGY: Assertion-Driven Invariant Synthesis
1. Decompose the assertion: each conjunct of the assert
   (split on &&) is a candidate loop invariant.
2. Add boundary invariants: weaken the loop guard.
   - while(i < n) needs: i <= n
   - while(n <= a) needs: n <= a+1
3. Preserve parameter values: if a parameter is unmodified,
   assert: loop invariant x == \at(x, Pre);
4. Verify sufficiency: check that (invariants AND NOT guard)
   implies the postcondition.

## FORBIDDEN CONSTRUCTS
- \forall, \exists  (quantifiers not supported)
- predicate, logic, axiomatic, lemma
- Ternary ? :
- \at(local_var, Pre)  (\at only on function parameters)
- ^ for exponentiation  (write x*x*x for x cubed)

## ALLOWED OPERATORS
Comparison: ==, !=, <, <=, >, >=
Logical:    &&, ||, !, ==>, <==>
Arithmetic: +, -, *, /, %

## OUTPUT
Return ONLY the annotated C function. Do not include
explanations.
\end{lstlisting}

\subsection{User Prompt (Loop Context Block)}

The user prompt is assembled by \texttt{PromptFormatter} and passed
as the second message in the conversation.
It consists of two parts.

\paragraph{Part 1 — Loop context.}
\begin{lstlisting}[style=coqstyle, basicstyle=\ttfamily\scriptsize]
### Loop Context: ###

1. Loop Context
  A. Pre-Condition (Before Loop Entry): `<precondition>`
  B. Loop Transition Relation: `<transition_expr>`
  C. Loop Snippet:
```c
<loop_code>
```

2. Execution Traces
   Each trace shows the full sequence of conditional
   evaluations, step by step.

   WARNING: CRITICAL TEMPORAL SEMANTICS:
   - Each state is labeled as
     'BEFORE loop body execution' or 'AFTER loop terminates'
   - Loop invariants must hold at the START of each iteration
     (BEFORE body executes)
   ...
\end{lstlisting}

\paragraph{Part 2 — Execution traces.}
For each sampled input group (up to $G{=}5$), the formatter emits
a trace block:
\begin{lstlisting}[style=coqstyle, basicstyle=\ttfamily\scriptsize]
[TRACE 1]
    [ n==0 && x==0 && y==1 && z==6 ]
        (BEFORE loop starts) ->
    [ n==1 && x==1 && y==7 && z==12 ]
        (BEFORE iteration 1 body executes) ->
    [ n==2 && x==8 && y==25 && z==18 ]
        (AFTER loop terminates)
\end{lstlisting}

The temporal labels (\texttt{BEFORE iteration $k$ body executes})
are essential: they tell the LLM that an invariant must hold at loop
entry of each iteration, not at loop exit, which is a common source
of error in naive prompting.
Initial values use $\texttt{\textbackslash at}(v, \text{Pre})$ notation
so the model can directly reference them in \texttt{loop invariant} lines.

\paragraph{Diversity via temperature.}
Each of the $N$ parallel LLM calls uses an independently sampled
temperature $\tau_j$.
In the default configuration, the primary call uses $\tau_1 = 1.0$ and
subsequent calls are drawn uniformly from $\{0.8, 1.0, 1.1, 1.2\}$,
ensuring that the candidate pool covers both conservative and more
exploratory invariant conjectures.

%% ============================================================
\section{Heuristic ACSL Syntax Gate: Implementation Details}
\label{app:syntax-gate}
%% ============================================================

The syntax gate (Stage~1, Section~\ref{sec:filtering}) runs before
any program execution or Frama-C invocation.
It is implemented in \texttt{unified\_filter.py} as a rule-based
checker that operates directly on ACSL expression strings.

\subsection{Rejection Rules}

Each candidate invariant expression is checked against
the following ordered rules (any single failure causes rejection):

\begin{enumerate}

\item \textbf{Trivial/placeholder tokens.}
  Expressions equal to \texttt{true}, \texttt{false}, \texttt{...},
  or containing only punctuation are rejected immediately.

\item \textbf{Non-printable or non-ASCII characters.}
  Frama-C requires printable ASCII (U+0020--U+007E).
  Characters such as the Unicode inequality signs \textsf{$\leq$}, \textsf{$\geq$},
  \textsf{$\neq$}, or control characters (e.g.\ BEL, U+0007 occasionally
  emitted by LLMs) cause rejection.
  The error message instructs the model to use \texttt{<=}, \texttt{!=}.

\item \textbf{Nested \texttt{loop invariant} keyword.}
  When a LLM fills in a \texttt{PLACE\_HOLDER} slot with a complete
  \texttt{loop invariant expr;} line instead of a bare expression,
  Frama-C cannot parse the annotation and enters a pathological parsing
  state.
  A regex check (\texttt{/\textbackslash bloop\textbackslash sinvariant\textbackslash b/i})
  catches and rejects these.

\item \textbf{Bracket balance.}
  Unmatched parentheses, square brackets, or braces are rejected,
  with the position of the first mismatch reported in the error message.

\item \textbf{Forbidden quantifiers.}
  Presence of \texttt{\textbackslash forall} or \texttt{\textbackslash exists}
  triggers rejection; Frama-C/WP does not support quantifiers in
  loop invariant annotations.

\item \textbf{Forbidden ACSL definitions.}
  The keywords \texttt{predicate}, \texttt{inductive}, \texttt{logic},
  \texttt{axiomatic}, \texttt{lemma} may not appear in invariant expressions.

\item \textbf{Forbidden ACSL math operators.}
  \texttt{\textbackslash product}, \texttt{\textbackslash sum},
  \texttt{\textbackslash min}, \texttt{\textbackslash max} are rejected
  (not supported by the WP plugin for loop invariants).

\item \textbf{Ternary operator.}
  The \texttt{?\ :} construct is not valid ACSL.

\item \textbf{Undefined variable names.}
  Each identifier in the expression is compared against
  $\mathit{known\_vars}$, the union of local variables, parameters,
  and ACSL built-ins extracted from $\mathbf{record}$.
  Unknown identifiers are rejected with a diagnostic listing the
  candidates.

\item \textbf{\texttt{\textbackslash at} scope check.}
  Expressions of the form $\texttt{\textbackslash at}(v, \text{Pre})$
  are permitted only when $v \in \mathcal{P}$ (function parameters).
  Usage on local variables is rejected, since local variables have no
  pre-state binding in Frama-C/WP.

\item \textbf{Exponentiation (\texttt{\^{}} or \texttt{\textbackslash pow}).}
  A final safety sweep removes any surviving candidates containing
  these tokens, which represent bitwise XOR or undefined ACSL
  functions respectively.

\end{enumerate}

\subsection{Post-Merge Sanitisation}

After the union of $N$ candidate pools is formed (Algorithm~\ref{alg:full},
line~11), an additional normalisation pass is applied:
\begin{itemize}
\item \texttt{\textbackslash at}($v$, Pre) for local variable $v$ is
  rewritten to plain $v$ (conservative safe fallback).
\item Any invariant re-introducing \texttt{\^{}} or \texttt{\textbackslash pow}
  is silently dropped.
\end{itemize}
This sanitisation step prevents the Houdini stage from encountering
annotation parse errors that would stall the Frama-C subprocess.

%% ============================================================
\section{Smart Dynamic Sampling: Implementation Details}
\label{app:sampling}
%% ============================================================

\subsection{Overview}

The smart sampler (\texttt{smart\_sampler.py}) generates concrete
integer inputs for the program's function parameters.
Its design goal is to produce a small but informationally dense set
of execution traces that expose the polynomial relationships maintained
by the loop, while avoiding:
\begin{itemize}
\item All-zero inputs (which collapse polynomial terms and prevent
  coefficient identification).
\item Excessively large inputs (which cause integer overflow or
  time-outs during dynamic execution).
\end{itemize}

\subsection{Tiered Value Generation}

For each parameter $v$ with domain $[\ell_v, u_v]$ inferred by
\texttt{LoopBoundAnalyzer}, the sampler constructs four tiers:
\begin{align*}
\mathcal{T}_0(v) &= \{0, 1, -1\} \cap [\ell_v, u_v]
  && \text{(special corner values)} \\
\mathcal{T}_1(v) &= \{2,\ldots,10\} \cap [\ell_v, u_v]
  && \text{(small positive integers)} \\
\mathcal{T}_{1'}(v) &= \{-2,\ldots,-10\} \cap [\ell_v, u_v]
  && \text{(small negatives, if enabled)} \\
\mathcal{T}_2(v) &\subseteq \{11,\ldots,50\}
  && \text{(5 randomly drawn medium values)} \\
\mathcal{T}_3(v) &\subseteq \{51,\ldots,100\}
  && \text{(3 randomly drawn large values)}
\end{align*}

\subsection{Phased Cartesian Sampling}

Inputs are generated in four phases.
Each phase fills part of the quota of at most $S_{\max}$
samples (default $S_{\max} = 20$).
Already-seen input tuples are deduplicated via a hash set.

\begin{enumerate}
\item \textbf{Phase~1 (corner cases).}
  Generate all tuples in $\prod_v \mathcal{T}_0(v)$.
  For $p$ parameters each with up to 3 special values, this produces
  at most $3^p$ tuples (e.g.\ 9 for $p=2$).
  These inputs are ideal for fitting degree-0 and degree-1
  polynomial terms.

\item \textbf{Phase~2 (mixed tier 0+1).}
  Enumerate combinations where each coordinate is drawn from
  $\mathcal{T}_0(v) \cup \mathcal{T}_1(v)$.
  Combinations are ordered by total tier index
  $\sum_v \text{tier}(v_j)$, so simpler combinations are generated
  first.

\item \textbf{Phase~3 (all tiers).}
  Extend to include $\mathcal{T}_2$ and $\mathcal{T}_3$,
  again prioritised by total tier index.
  This covers medium-range inputs needed to disambiguate
  quadratic from linear relationships.

\item \textbf{Phase~4 (biased random fill).}
  If the quota is not yet reached, fill with random tuples
  where each coordinate is drawn with probability 0.8 from
  $[0, 20]$ and 0.2 from $[21, 100]$.
  This phase ensures coverage of edge cases not reached
  by the structured phases.
\end{enumerate}

\subsection{Trace Collection and Formatting}

Each accepted input tuple is passed to
\texttt{DynamicExecutorConfigurable}, which instruments the C source
with \texttt{printf} statements recording all variable values at the
top of each loop iteration.
The collected trace for a single input is a sequence of
variable-assignment strings (e.g.
\texttt{(n == 2) * (x == 8) * (y == 25) * (z == 18)}).
These are parsed and formatted by \texttt{PromptFormatter}
into the temporal-labelled trace blocks shown in
Appendix~\ref{app:prompts}.

To prevent context-window overflow, the formatter applies two
limits:
\begin{itemize}
\item At most $G = 5$ trace groups are included per LLM call.
\item Within each group, at most $k_{\max} = 10$ iteration
  snapshots are kept; if a loop runs for more than $k_{\max}$
  iterations, the first $\lfloor k_{\max}/2 \rfloor$ and last
  $\lceil k_{\max}/2 \rceil$ snapshots are retained, and the
  truncation is noted explicitly in the prompt.
\end{itemize}
These limits ensure that the total prompt length stays within
the LLM's context window even for programs with deeply nested
loops or large parameter ranges.

%% ============================================================

\bibliographystyle{ACM-Reference-Format}
\bibliography{references}

\end{document}
